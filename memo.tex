\documentclass{article}

\usepackage[margin=1in]{geometry} % reduce the margin
\usepackage{expl3} % for variable length list

% Disable indentation at the beginning of each paragraph
\setlength{\parindent}{0pt}

% user-defined commands for note entries
\ExplSyntaxOn
\NewDocumentCommand{\CommandEntry}{mm}
{
    \texttt{#1}
    \begin{itemize}
        \seq_set_split:Nnn \l_tmpa_seq { ; } { #2 } % set semicolon as list item separator
        \seq_map_inline:Nn \l_tmpa_seq { \item ##1 }
    \end{itemize}
    \vspace{1em}
}
\ExplSyntaxOff

\ExplSyntaxOn
\NewDocumentCommand{\FileEntry}{mm}
{
    \texttt{#1}
    \begin{itemize}
        \seq_set_split:Nnn \l_tmpa_seq { ; } { #2 } % set semicolon as list item separator
        \seq_map_inline:Nn \l_tmpa_seq { \item ##1 }
    \end{itemize}
    \vspace{1em}
}
\ExplSyntaxOff

\ExplSyntaxOn
\NewDocumentCommand{\EnvVarEntry}{mm}
{
    \texttt{#1}
    \begin{itemize}
        \seq_set_split:Nnn \l_tmpa_seq { ; } { #2 } % set semicolon as list item separator
        \seq_map_inline:Nn \l_tmpa_seq { \item ##1 }
    \end{itemize}
    \vspace{1em}
}
\ExplSyntaxOff

\ExplSyntaxOn
\NewDocumentCommand{\ScriptEntry}{mm}
{
    \textbf{#1}
    \begin{itemize}
        \seq_set_split:Nnn \l_tmpa_seq { ; } { #2 } % set semicolon as list item separator
        \seq_map_inline:Nn \l_tmpa_seq { \item ##1 }
    \end{itemize}
    \vspace{1em}
}
\ExplSyntaxOff

\begin{document}

\section{Accessibility}

    \CommandEntry{setterm -inversescreen on}
    {
        Set console terminal background to white and font to black
    }

    \CommandEntry{setterm -inversescreen off}
    {
        Revert the previous command.
    }

    \CommandEntry{setterm -background <color>}
    {
        Set console terminal background to \texttt{<color>};
        \texttt{<color>} can be black, red, green, yellow, blue, magenta, cyan, and white.
    }

    \CommandEntry{setterm -foreground <color>}
    {
        Set console terminal font to \texttt{<color>};
        \texttt{<color>} can be black, red, green, yellow, blue, magenta, cyan, and white.
    }

    \CommandEntry{setterm -reset}
    {
        Change the terminal appearance back to the default settings and clear the screen.
    }

    \CommandEntry{setterm -store}
    {
        Store the current appearance as the default settings.
    }

\section{Helpers}

    \CommandEntry{man <topic>}
    {
        Retrieve help information about \texttt{<topic>} from the Linux manual.
    }

    \CommandEntry{man <section \#> <topic>}
    {
        Retrieve help information about \texttt{<topic>} in section \texttt{<section \#>} from the Linux manual.
    }

    \CommandEntry{info <topic>}
    {
        Retrieve help information about \texttt{<topic>} from info page.
    }

    \CommandEntry{help <topic>}
    {
        Retrieve help information about \texttt{<topic>}. \texttt{<topic>} should be a built-in command.
    }

\section{Navigating the Filesystem}
     
    \CommandEntry{cd}
    {
        Come back to the home directory.
    }

    \CommandEntry{pwd}
    {
        Display the current working directory.
    }

    \CommandEntry{ls <filter>}
    {
        Filter the output to match the pattern specified by \texttt{<filter>} Wildcards are applicable here.
    }

    \CommandEntry{ls -l}
    {
        Long listing.;
        \textbf{Note:} apply \texttt{ls -l} to a symbolic link of a directory won't list the files within the underlying directory. However, apply \texttt{ls} will.
    }

    \CommandEntry{ls -F}
    {
        Append a slash (\texttt{/}) to the directories for better distinguishability.
    }

    \CommandEntry{ls -R}
    {
        Traverse and print the listing of the current directory recursively.
    }

    \CommandEntry{ls -d <dir\_path>}
    {
        Display the info of \texttt{<dir\_path>} instead of peek into the \texttt{<dir\_path>}
    }

    \CommandEntry{ls -l --time=atime}
    {
        Display the access time instead of modification time in the long listing.
    }

    \CommandEntry{ls -i}
    {
        Display \textit{inode} number.
    }

\section{Handling Files}
    
    \CommandEntry{touch -a <file>}
    {
        Change the access time of \texttt{<file>} to the current time.
    }

    \CommandEntry{touch <existing\_file>}
    {
        Change the modification time of \texttt{<file>} to the current time.
    }

    \CommandEntry{cp <src> <dst>}
    {
        Copy \texttt{<src>} to \texttt{<dst>};
        Wildcards can be used on \texttt{<src>} to copy multiple files.
    }

    \CommandEntry{cp -i <src> <dst>}
    {
        Copy \texttt{<src>} to \texttt{<dst>};
        Inform the user before overwriting.;
        \textbf{Highly recommended usage.}
    }

    \CommandEntry{ln -s <src> <sym\_link>}
    {
        Create a symbolic link \texttt{<sym\_link>} pointing to \texttt{<src>};
        \textbf{Do not copy symbolic links, which can easily result in confusion}
    }

    \CommandEntry{ln <src> <hard\_link>}
    {
        Create a hard link \texttt{<hard\_link>} pointing to \texttt{<src>};
        \textbf{Do not copy hard links, which can easily result in confusion};
        \textbf{You cannot create hard links for the files residing on a different physical media.}
    }

    \CommandEntry{mv <src> <dst>}
    {
        Move \texttt{<src>} to \texttt{<dst>};
        "Move" is a fancy way to say "rename" in the Linux world.;
        \texttt{mv} can only change file names. It cannot change inode numbers or timestamps of files.
    }

    \CommandEntry{mv -i <src> <dst>}
    {
        Move \texttt{<src>} to \texttt{<dst>} Inform the user before overwriting.;
        \textbf{Highly recommended usage.}
    }

    \CommandEntry{rm -i <file>}
    {
        Inform users before removing \texttt{<file>};
        \textbf{Highly recommended usage.}
    }

    \CommandEntry{rmdir <empty\_dir>}
    {
        Remove the empty directory \texttt{<empty\_dir>};
        Fail when the specified directory is not empty.
    }

    \CommandEntry{tree <dir>}
    {
        Display the structure of the specified directory.;
        \texttt{tree} utility may not installed by default.
    }

    \CommandEntry{find <dir> -iname <pattern>}
    {
        Find occurences of case insensitive \texttt{<pattern>} in the directory \texttt{<dir>};
        \texttt{<pattern>} can be regex to match file names.
    }

\section{Peeking File Contents}

    \CommandEntry{file <file>}
    {
        Check the type of the file.;
        When the \texttt{<file>} is an executable binary file, \texttt{file} command can provide useful information such as target platform and required libraries.
    }

    \CommandEntry{cat -n <file>}
    {
        Display the full contents with line numbers listed at LHS.
    }

    \CommandEntry{cat -b <file>}
    {
        Display the full contents with line numbers listed only for non-empty lines.
    }

    \CommandEntry{cat -T <file>}
    {
        Replace Tabs with \texttt{\string^I}
    }

    \CommandEntry{zcat <gz\_file>}
    {
        Display the contents of a text file contained in the gz archive.
    }

    \CommandEntry{more <file>}
    {
        Display the file content part by part.;
        It's a pager utility, so it overcomes the drawback of \texttt{cat} command, which may quickly flush the whole screen.;
        If some commands produce huge amount of data, they can be piped to \texttt{more} to achieve page-wise reading.;
        \textbf{Example:} \texttt{dmesg | more}
    }

    \CommandEntry{less <file>}
    {
        A more powerful utility than \texttt{more};
        \textit{Less is more}
    }

    \CommandEntry{head -n <\#> <file>}
    {
        Display the first \texttt{<\#>} lines of the file.;
        Shorthand form: \texttt{head -<\#> <file>};
        By default, \texttt{head} display the first 10 lines.
    }

    \CommandEntry{tail -n <\#> <file>}
    {
        Display the last \texttt{<\#>} lines of the file.;
        Shorthand form: \texttt{tail -<\#> <file>};
        By default, \texttt{tail} display the first 10 lines.
    }

    \CommandEntry{tail -f <file>}
    {
        Display the last 10 lines of the file in real-time.
    }

\section{Monitor Processes}

    \CommandEntry{ps -ef}
    {
        See all processes running on the system.;
        Not real-time monitoring. It's just a snapshot.;
        \texttt{-ef} are UNIX style parameters.
    }

    \CommandEntry{ps -l}
    {
        Display more information.
    }

    \CommandEntry{ps l}
    {
        Alternative to \texttt{ps -l};
        Give more detailed information about status code.;
        \texttt{l} is a BSD style parameter.
    }

    \CommandEntry{ps --forest}
    {
        Display the hierarchical structure of the processes.;
        Shorthand form: \texttt{ps f};
        \texttt{--forest} is a GNU style parameter.
    }

    \CommandEntry{top}
    {
        Monitor processes in real-time.;
        An interactive utility. Press \texttt{f} to select the field used for sorting, \texttt{d} to change the polling interval, and \texttt{q} to quit.
    }

    \CommandEntry{sudo kill <PID>}
    {
        Send \texttt{TERM} signal to the process for possible termination.;
        Must use PID instead of the name of a process.
    }

    \CommandEntry{sudo kill -s <SIGNAL> <PID>}
    {
        Send signal \texttt{<SIGNAL>} to the process.;
        \texttt{<SIGNAL>} can be either the number or the name of the signal.;
        For forcible termination, use \texttt{KILL} signal.
    }

    \CommandEntry{sudo killall <PNAME>}
    {
        Stop the processes by their names. Wildcards is applicable in \texttt{<PNAME>}
    }

\section{System Services Management}

    \CommandEntry{sudo systemctl enable <service>}
    {
        Enable \texttt{<service>} to automatically run at the boot time.
    }

    \CommandEntry{sudo systemctl disable <service>}
    {
        Disable \texttt{<service>} from starting at the boot time.;
        Reverse operation of \texttt{sudo systemctl enable <service>}
    }

    \CommandEntry{sudo systemctl start <service>}
    {
        Start a stopped service \texttt{<service>}
    }

    \CommandEntry{sudo systemctl stop <service>}
    {
        Stop a running service \texttt{<service>};
        Reverse operation of \texttt{sudo systemctl start <service>}
    }

    \CommandEntry{sudo systemctl reload <service>}
    {
        Reload \texttt{<service>} after modify its configuration files.
    }

    \CommandEntry{sudo systemctl restart <service>}
    {
        Restart a service.
    }

    \CommandEntry{sudo systemctl status <service>}
    {
        Display the status of a service.
    }

\section{Manage Secondary Storage Media}

    \CommandEntry{mount}
    {
        List all mounting media.
    }

    \CommandEntry{mount -t <filesys\_type> <dev\_file> <mount\_dir>}
    {
        Mount the device specified by \texttt{<dev\_file>} to the directory \texttt{<mount\_dir>};
        \texttt{<filesys\_type>} is the type of the file system residing on the device.;
        \textbf{Example: Creating RAM Disk} \\
        \texttt{mount -t tmpfs -o size=512M,noswap ramdisk /mnt/ramdisk}
        \begin{itemize}
            \item using RAM-based \texttt{tmpfs} file system
            \item mounting point is \texttt{/mnt/ramdisk}
            \item name \texttt{ramdisk} and size \texttt{512M}
            \item swap functionality built in \texttt{tmpfs} is disabled
            \item \texttt{size} and \texttt{noswap} can be found in \texttt{man 5 tmpfs}
        \end{itemize}
    }

    \CommandEntry{umount <mount\_dir | dev\_file>}
    {
        Unmount a mounted media by either the mounting directory or device file name.;
        The name of the command is \texttt{umount}, not \texttt{unmount};
        If the media is using by some processes, unmount will fail.
    }

    \CommandEntry{df}
    {
        Check the usage of disks.
    }

    \CommandEntry{df -h}
    {
        Show the disk space in human-readable form.
    }

    \CommandEntry{du <dir\_list>}
    {
        Show the disk usage for specific directories in a recursive manner.;
        If \texttt{<dir\_list>} is missing, current directory will be the default.;
        \texttt{-s} output the size for each directory listed. (summary functionality);
        \texttt{-c} will give a grand total result for all directories at the end of the output.;
        \texttt{-h} will convert the space into human-readable form.
    }

    \CommandEntry{dd if=/dev/zero of=<output\_file> bs=1M count=256 conv=fdatasync}
    {
        Test the write speed of a file system.;
        \texttt{<output\_file>} should be located in the file system to be tested.;
        \texttt{bs} is block size.;
        \texttt{count} is the number of the blocks to be written.;
        \texttt{conv=fdatasync} will ensure the data have been physically written to the storage media before the program returns.
    }

    \CommandEntry{dd if=<input\_file> of=/dev/null bs=1M}
    {
        Test the read speed of a file system.;
        \texttt{<input\_file>} should be located in the file system to be tested.;
        \texttt{conv=fdatasync} is not necessary because no writing is involved.;
        Read means copying data to the main memory from the storage media.
    }

\section{Manipulating Data Files}

    \CommandEntry{sort <file>}
    {
        Sorting lines in dictionary order.;
        Ascending order by default.
    }

    \CommandEntry{sort -c <file>}
    {
        Check whether \texttt{<file>} is already sorted.
    }

    \CommandEntry{sort -r <file>}
    {
        Sort in descending order.
    }

    \CommandEntry{sort -n <file>}
    {
        Treat data as numeric data.
    }

    \CommandEntry{sort -M <file>}
    {
        Sort in month order.;
        Months must be in three-character format, such as Jan, Feb and Mar.
    }

    \CommandEntry{sort -t <delimiter> -k <pos> <file>}
    {
        \texttt{-t}: split each line with \texttt{<delimiter>};
        \texttt{-k}: after spliting, sort according to \texttt{<pos>}-th column.;
        \texttt{-k} can appear multiple times to sort a joint key. For example, \\
        \texttt{sort -t, -k1,1n -k2,2n input.csv} \\
        The first key starts from column 1 and ends at column 1, and is treated as a numeric key. \\
        The second key starts from column 2 and ends at column 2, and is treated as a numeric key.
    }

    \CommandEntry{cut -d <delimiter> -f <number> <file>}
    {
        Process \texttt{<file>} line by line. Split each line into fields with the \texttt{<delimiter>} Only preserve the \texttt{<number>}-th (one-based) field to output.;
        In order to preserve multiple lines, treat \texttt{<number>} as a comma-separated list.
    }

    \CommandEntry{grep -v <pattern> <file>}
    {
        Display the lines which do not have the matching occurences. 
    }

    \CommandEntry{grep -n <pattern> <file>}
    {
        Display line numbers.
    }

    \CommandEntry{grep -c <pattern> <file>}
    {
        Display the count of lines that match the pattern.
    }

    \CommandEntry{grep -e <pattern\_1> -e <pattern\_2> \dots\ -e <pattern\_n> <file>}
    {
        Specify multiple patterns to match.
    }

    \CommandEntry{gzip <file1> <file2> <file3> \dots <filen>}
    {
        Compress files \textbf{in place}.;
        Wildcards can be used in \texttt{<file1> <file2> <file3> \dots <filen>}.;
        \texttt{<file1> <file2> <file3> \dots <filen>} will be deleted after compression.;
        Another compression utility, \texttt{xz}, has a higher compression ratio but similar command syntax as \texttt{gzip}
    }

    \CommandEntry{gzip -d <gzip\_archive1> <gzip\_archive2> \dots <gzip\_archiven>}
    {
        Decompress a gzip archive \textbf{in place}.;
        \texttt{<gzip\_archive1> <gzip\_archive2> \dots <gzip\_archiven>} will be deleted after decompression.;
        Equivalent to \texttt{gunzip <gzip\_archive1> <gzip\_archive2> \dots <gzip\_archiven>} while \texttt{gunzip} is a shell script which utilizes \texttt{gzip -d}
    }

    \CommandEntry{gzip -r <dir>}
    {
        Descend into \texttt{<dir>} recursively and compress files \textbf{in place}.;
        \texttt{-r} alone should be interpreted as "perform the task recursively".;
        Will not archive the whole directory structure into the archive. So, use \texttt{tar} to archive the directory first before compression.;
        Combining with \texttt{-d} option will descend into \texttt{<dir>} recursively and decompress files \textbf{in place}.
    }

    \CommandEntry{gzip -v <file1> <file2> <file3> \dots <filen>}
    {
        Verbose mode. Print old file names, new file names, and compression ratio.;
        \texttt{-v} alone should be interpreted as "perform the task in verbose mode".;
        Can be used with \texttt{-d} and \texttt{-r} etc.
    }

    \CommandEntry{gzip -l <gzip\_archive1> <gzip\_archive2> \dots <gzip\_archiven>}
    {
        List the detailed information about \texttt{<gzip\_archive1> <gzip\_archive2> \dots <gzip\_archiven>};
        Can be used with \texttt{-r}
    }

    \CommandEntry{gzip -k <file1> <file2> <file3> \dots <filen>}
    {
        Compress files but keep the original files after completion.;
        \texttt{-k} alone should be interpreted as "keep the original files after completion".;
        Can be used with \texttt{-d}
    }

    \CommandEntry{gzip -<\#> <file1> <file2> <file3> \dots <filen>}
    {
        Set compression level, which can be 1 to 9. Higher the level, larger the compression ratio, slower the speed.;
        \texttt{-1} is equivalent to \texttt{--fast} and \texttt{-9} is equivalent to \texttt{--best}
    }

    \CommandEntry{gzip -f <file1> <file2> <file3> \dots <filen>}
    {
        If files with the same names exist, overwrite after compression.;
        Pretty useful if we want to run in non-interactive mode.;
        Can be used with \texttt{-d}
    }

    \CommandEntry{gzip --suffix <new\_suffix> <file1> <file2> <file3> \dots <filen>}
    {
        Use the specified suffix instead of the default one for compression.;
        \texttt{<new\_suffix>} should be preceded by a dot (.) for the expected effect.
    }

    \CommandEntry{tar -cvf <archive.tar> <object1> <object2> \dots <objectn>}
    {
        Archive all listed objects into the tar archive.;
        Objects can be either files or directories.;
        Unlike \texttt{gzip}, \texttt{tar} won't delete the original files archived. 
    }

    \CommandEntry{tar -tf <archive.tar>}
    {
        Only list the files of the tar archive.
    }

    \CommandEntry{tar -xvf <archive.tar>}
    {
        Extract everything from the tar archive.;
        If the archive is compressed, such as \texttt{tar.gz} and \texttt{tar.xz}, this command also works. \texttt{tar} can automatically identify the format of the file and extract the files.;
        In \texttt{tar -Jxvf <archive.tar.bz2>}, \texttt{-J} option is used to filter out \texttt{bz2} compressed file.
    }

    \CommandEntry{tar -zxvf <archive.tgz>}
    {
        Extract everything from a gzipped tar archive.;
        \texttt{<archive.tgz>} can also be \texttt{<archive.tar.gz>}
    }

\section{Shell}

    \subsection{Subshell}

        \CommandEntry{bash <parameters>}
        {
            \texttt{<parameters>} can be: \\
            \texttt{-c command} run \texttt{command} in the child shell. \\
            \texttt{-i} start interactive shell. \\
            \texttt{-l} start a login-like shell. \\
            \texttt{-r} run restricted shell which limits users to the default directory. \\
            \texttt{-s} read commands from standard input.
        }

        \CommandEntry{( command1 ; command2 ; \dots ; commandn )}
        {
            Process list, a command grouping type. Run commands one by one in the subshell.;
            \texttt{( command1 ; command2 ; ( command 3 ) )} will create a grandchild shell.;
            Expensive method for multiprocessing. Not truly multiprocessing.;
            \texttt{\{ command1; command2; \dots; commandn; \}} is another command grouping type but won't create subshell.;
            \texttt{command1 ; command2 ; \dots ; commandn} will run commands one by one without starting a subshell.
        }

        \CommandEntry{<command>\&}
        {
            Put and run the command in background mode.;
            Press Enter key to let the prompt come back.;
            \texttt{[<\#\_of\_background\_jobs>] <PID>} will be given before the prompt returns.;
            \textbf{No} subshell is spawned.
        }

        \CommandEntry{<process\_list>\&}
        {
            Put the and run the process list in the background mode.;
            Subshell's I/O is not tied to the terminal.
        }

        \CommandEntry{jobs -l}
        {
            Display the detailed information about the background jobs.
        }

        \CommandEntry{coproc <command>}
        {
            Co-processing. Spawn a subshell in background mode and execute \texttt{<command>} within the subshell.;
            Use extended syntax to specify the job name: \texttt{coproc <job\_name> \{ <command>; \}}
        }

        \CommandEntry{coproc <process\_list>}
        {
            Combine co-processing with process list.
        }

    \subsection{About Built-in Commands}

        \CommandEntry{which <command\_driver>}
        {
            Find the file location of a command.;
            Only works for external commands.
        }

        \CommandEntry{type -a <command\_driver>}
        {
            Find all file locations of a command.;
            \texttt{type} can also tell you whether a command is built-in or not.;
            Some commands can have both flavor, which can be revealed by \texttt{type -a};
            To use an external command which has multiple flavors, directly reference the file. For example, \texttt{/usr/bin/pwd}
        }

        \CommandEntry{sudo -i}
        {
            Allow sudoers login as root in the current terminal seesion without needing root's password.;
            \texttt{sudo su} also works.
        }

        \CommandEntry{history}
        {
            Show commands used recently in the bash shell.;
            Commmand history of bash is stored in \texttt{/home/<username>/.bash\_history}.;
            Command history is first written to the memory, then saved to the file after bash is exited.;
            \texttt{.bash\_history} is only read when a bash session first starts.;
            Edit the environment variable \texttt{HISTSIZE} to change the number of records that will be kept.;
            Each bash session maintain their own history in the memory.;
            \texttt{history -a} will append bash history in the memory to \texttt{.bash\_ history};
            \texttt{history -n} will reread \texttt{.bash\_history}
        }

        \CommandEntry{!!}
        {
            Recall and execute the previous command.
        }

        \CommandEntry{!<\#>}
        {
            Pull out the \texttt{<\#>}-th history and execute.
        }

        \CommandEntry{alias -p}
        {
            See the list of aliases.
        }

        \CommandEntry{alias <new\_command>='<definition>'}
        {
            Create new aliases.;
            Since \texttt{alias} is a built-in, new command is only active in the session where it is defined.;
            Place aliases into \texttt{.bashrc} to make them persistent.
        }

    \subsection{Environment Variables}

        \CommandEntry{env}
        {
            Print all system environment variables.
        }

        \CommandEntry{printenv}
        {
            Print all system environment variables.;
            \texttt{printenv <var\_name>} print the value of the specified \textbf{system} variable.
        }

        \CommandEntry{set}
        {
            Display all variables defined for a specific process.
        }

        \CommandEntry{<var\_name>=<value>}
        {
            Define or change a local environment variable.;
            Local Variables are visible only in the shell session where it was defined.;
            No whitespaces.;
            \texttt{<value>} can be either numeric or a string.;
            \texttt{<var\_name>} should be in lowercase if it's a user-defined variable.;
            \texttt{<var\_name>} should be in uppercase if it's a system variable.
        }

        \CommandEntry{unset <var\_name>}
        {
            Remove an environment variable.
        }

        \CommandEntry{export <var\_name>}
        {
            Set an existing local variable to be global.;
            Global variables defined in a parent shell is visible to the child shell. Converse is not true.;
            Any modifications to the global variables done by a child shell will not affect the global variables in the parent shell.
        }

        \CommandEntry{PATH=\$PATH:<new\_location>}
        {
            Append new locations to the \texttt{PATH} environment variable.;
            Remember to \texttt{export} the new \texttt{PATH} variable.;
            Changes can only last until we exit the system or reboot the system.
        }

        \CommandEntry{\$<var\_name>}
        {
            Reference to a variable.;
            \texttt{\$\{<var\_name>\}} also works.
        }

        \CommandEntry{<array\_name>=(<item1> <item2> \dots <itemn>)}
        {
            Variables array definition.
        }

        \CommandEntry{\$\{<array\_name>[<index>]\}}
        {
            Reference to an element.;
            \texttt{<index>} is zero-based.;
            \texttt{\$\{<array\_name>\}} only gives you the first element.
        }

        \CommandEntry{\$\{<array\_name>[*]\}}
        {
            Get the entire array.
        }

        \CommandEntry{\$\{<array\_name>[<index>]=<new\_value>\}}
        {
            Change the value of an element.
        }

        \CommandEntry{unset <array\_name>}
        {
            Remove the entire array.
        }

        \CommandEntry{unset <array\_name>[<index>]}
        {
            Remove a certain element from the array.;
            \textbf{Warning:} remove operation won't result in shifting elements to the left.
        }

\section{Permission Management}

    \subsection{User Management}

        \CommandEntry{sudo useradd -m <login\_name>}
        {
            Create new user and set up home directory (\texttt{-m} option).;
            It uses some default values stored in \texttt{/etc/default/useradd} to create the new user.
        }

        \CommandEntry{sudo useradd -D}
        {
            Peek the system default used for \texttt{useradd}.;
            Admin can place template files within the directory specified by \texttt{SKEL} field.;
            Add one more parameter can override the system default.
        }

        \CommandEntry{sudo userdel <username>}
        {
            Remove a user from the system.;
            By default, only the information in \texttt{/etc/passwd} is removed.;
            Add \texttt{-r} to remove user's home directory and mail directory (\textbf{not recommended}, because files in the home directory may be still in use by other users), but this still won't remove other files that belong to the user.
        }

        \CommandEntry{sudo usermod}
        {
            Modify a user's account.;
            \texttt{-c} change the comment field.;
            \texttt{-e} change the expiration date.;
            \texttt{-g} change the default login group.;
            \texttt{-l} change the login name.;
            \texttt{-L} lock the account to prevent the user from login.;
            \texttt{-U} unlock the account.;
            \texttt{-p} change the password.
        }

        \CommandEntry{sudo passwd <username>}
        {
            Change a user's password.;
            A missing \texttt{<username>} will instruct the command to use the current user by default.;
            \texttt{-e} will force the user to change their password on the next login.
        }

        \CommandEntry{sudo chpasswd < <list.txt>}
        {
            Change users' password in bulk.;
            \texttt{<list.txt>} is a \texttt{<user\_id>:<new\_password>} list separated by newlines.
        }

        \CommandEntry{sudo chsh -s <path\_to\_new\_shell> <username>}
        {
            Change the default login shell for a user.
        }

        \CommandEntry{sudo chfn <username>}
        {
            Change the comment field of a user.;
            The comment is in the finger format.
        }

        \CommandEntry{finger <username>}
        {
            Display the information about a user on the system.
        }

        \CommandEntry{chage}
        {
            Manage the aging process of a user's password.;
            \texttt{-d} set the number of days since the password was last changed.;
            \texttt{-E} set the date the password expires. The format of the date can be either in \texttt{YYYY-MM-DD} or the number of days since Jan 1, 1970.;
            \texttt{-I} set the number of inactive days to lock the account after the password expires.;
            \texttt{-m} set the minimum number of days between the password changes.;
            \texttt{-W} set the number of days before the password expires to give the user a warning.
        }

    \subsection{Group Management}

        \CommandEntry{groupadd <group\_name>}
        {
            Add a new group.
        }

        \CommandEntry{usermod -G <group\_name> <username>}
        {
            Add the user with \texttt{username} to the group with \texttt{group\_name};
            If the membership of a user who is currently logged in is changed, log out that user and log in again to make the change take effect.;
            \texttt{-G} won't change a user's default group. \texttt{-g} is designed to change a user's default group.
        }

        \CommandEntry{groupmod -n <new\_name> <old\_name>}
        {
            Change the group name.;
            GID and group members will remain intact.
        }

        \CommandEntry{groupmod -g <new\_id> <group\_name>}
        {
            Change the GID of a group.
        }
        
        \CommandEntry{id -g <username>}
        {
            Display active group IDs of the specified user.;
            If \texttt{<username>} is ignored, user of the current process will be used.;
            Combine with \texttt{-n} option to convert group IDs to group names.;
            Use \texttt{-G} to display all IDs of groups that the user belongs to.
        }

    \subsection{File Permission}

        \CommandEntry{umask}
        {
            Display the default permission assigned to the new file or directory.;
            The first digit is the sticky bit.;
            The following 3 digits are the complement of octal mode security settings.;
            The full permission set for a new file is 666 and for a new directory is 777.
        }

        \CommandEntry{umask <new\_mask>}
        {
            Change the default value of \texttt{umask};
            \texttt{<new\_mask>} is the new complement of the octal mode security settings.
        }

        \CommandEntry{chmod <octal> <file>}
        {
            Set the permission of \texttt{<file>} to \texttt{<octal>};
            If \texttt{<file>} is a directory, apply \texttt{-R} will apply the settings to all subdirectories and files.;
            Wildcards can be used for \texttt{<file>} to match multiple file.;
            \textbf{Example:} \texttt{chmod 766 myprog} will set the permission of \texttt{myprog} to \texttt{rwxrw-rw-}
        }

        \CommandEntry{chmod [ugoa][+-=][rwxXstugo] <file>}
        {
            \texttt{ugoa} are owner, group, everyone else, and all above, respectively.;
            \texttt{+-=} means add, remove, and set permissions, respectively.;
            \texttt{s} means set UID (\texttt{u+s}) or GID (\texttt{g+s}).;
            If \texttt{[ugoa]} is omitted, \texttt{a} will be used as the default value.;
            \textbf{Example:} \texttt{chmod o+x myprog} will give everyone else execute permission.
        }

        \CommandEntry{sudo chown <new\_owner> <file>}
        {
            Change the owner to \texttt{<new\_owner>} for \texttt{<file>};
            \texttt{<new\_owner>} can be either a login name or UID.;
            Can also change the group simultaneously: \texttt{sudo chown <new\_owner>:<new\_group> <file>};
            A shorthand to change the group simultaneously if the distribution uses an individual group that match the user's login name: \texttt{sudo chown <new\_owner>: <file>};
            Use \texttt{-R} to apply changes recursively.;
            \textbf{Note:} If \texttt{<file>} is a symlink, the ownership of the symlink itself won't be changed. Instead, what will be changed is the ownership of its target. In order to change the ownership of a symlink instead of its target, apply \texttt{-h} option.;
            Only admins can change the owner of a file. Everyone can change the default group of a file, provided that they are in the group that the file leaves from and goes into.
        }

        \CommandEntry{chgrp <new\_group> <file>}
        {
            Change the default group of a file.;
            User must own the file and must be in \texttt{<new\_group>}
        }

\section{Manage Filesystem}

    \CommandEntry{sudo fdisk <path\_to\_dev\_file>}
    {
        Open a disk device with \texttt{fdisk} for further process.;
        Usually used to partition the disk.;
        Can also declare a partition as a part of LVM, which helps to change a physical partition into a physical volume.
    }

    \CommandEntry{sudo mkfs.ext4 <path\_to\_partition\_file>}
    {
        Format a partition with ext4 filesystem.
        \texttt{<path\_to\_partition\_file>} can be either a physical partition or a logical volume.
    }

    \CommandEntry{sudo fsck <option> <filesystem1> <filesystem2> \dots <filesystemn>}
    {
        Check and repair filesystems.;
        Filesystems in the command can be referenced by device file, mounting point, or UUID.;
        Filesystems shoule be unmounted before checking.
    }

    \CommandEntry{sudo pvcreate <path\_to\_partition\_file>}
    {
        Change a physical partition into a physical volume.;
        Before this, use \texttt{fdisk} to declare the partition as a part of the LVM.
    }

    \CommandEntry{sudo pvdisplay <path\_to\_partition\_file>}
    {
        Display the detailed information about the physical volume.
    }

    \CommandEntry{sudo vgcreate <new\_vol\_group\_name> <path\_to\_phy\_vol\_file>}
    {
        Create new volume group.
    }

    \CommandEntry{sudo vgdisplay <vol\_group\_name>}
    {
        Display the detailed information about the logical volume group.
    }

    \CommandEntry{vgchange}
    {
        Activate or deactivate a volume group.
    }

    \CommandEntry{vgremove}
    {
        Remove a volume group.
    }

    \CommandEntry{vgextend}
    {
        Add physical volumes to the volume group.
    }

    \CommandEntry{vgreduce}
    {
        Remove physical volumes from the volume group.
    }

    \CommandEntry{sudo lvcreate -l <percent> -n <lv\_name> <vol\_group\_name>}
    {
        Create logic volume.;
        \texttt{<percent>} is the percent of the free space that will be allocated to the logic volume.;
        \texttt{-L} allow you to specify the actual size (in KB, MB) that will be allocated to the logic volume.
    }

    \CommandEntry{vldisplay <vol\_group\_name>}
    {
        Display the detailed information about the logic volume.
    }

    \CommandEntry{lvextend}
    {
        Increase the size of a logical volume.
        \textbf{Note:} filesystems reside in the logic volume must be adjusted manually to fit the change.;
        Use \texttt{resize2fs} to reformat the filesystem.
    }

    \CommandEntry{lvreduce}
    {
        Reduce the size of a logical volume.
    }

\section{Install Softwares}

    \CommandEntry{apt --installed list}
    {
        List all installed packages on the machine.;
        \texttt{apt} is the front end of \texttt{apt-get} and \texttt{apt-cache}
    }

    \CommandEntry{apt show <package\_name>}
    {
        Only show the information about \texttt{package\_name};
        Cannot tell you whether the package is installed or not.
    }

    \CommandEntry{apt search <keyword>}
    {
        Display the packages match the keyword.;
        Search for the keyword in both package names and descriptions.;
        Use \CommandEntry{apt --names-only search <keyword>} to restrict the search in package names.
    }

    \CommandEntry{sudo apt install <package\_name>}
    {
        Install a package and its dependencies.
    }

    \CommandEntry{sudo apt update \&\& sudo apt upgrade}
    {
        Update the package repository and then upgrade all upgradable packages safely.;
        "Safely" means being free from dependency issues.;
        Unused packages after the upgrade won't be removed.;
        To remove unused packages, use \texttt{sudo apt full-upgrade} instead.;
        Upgrade should be performed regularly to get security updates in time.;
        Upgrade should be performed after a fresh installation to get security updates.
    }

    \CommandEntry{dpkg -L <package\_name>}
    {
        List all files related to the package.
    }

    \CommandEntry{dpkg --search <abs\_path\_to\_file>}
    {
        Find the package that the file belongs to.
    }

    \CommandEntry{sudo apt remove <package\_name>}
    {
        Remove a package with data and configuration files preserved.
    }

    \CommandEntry{sudo apt autoremove}
    {
        Remove used dependencies after removing a package.
    }

    \CommandEntry{sudo apt purge <package\_name>}
    {
        Remove the package with its data and configuration files.
    }

    \CommandEntry{flatpak list}
    {
        List all installed applications.
    }

    \CommandEntry{flatpak search <app\_id>}
    {
        Find an application from flatpak's repository.;
        Must use the application ID instead of the name when working with the container.
    }

    \CommandEntry{flatpak install <app\_id>}
    {
        Install an application.
    }

    \CommandEntry{flatpak uninstall <app\_id>}
    {
        Remove an application from the container.
    }

    \CommandEntry{./configure \\ make \\ make install}
    {
        Build and install a program from source code.
    }

\section{Networks}

    \CommandEntry{ssh-keygen -t <key\_type> -b <key\_length>}
    {
        Generate ssh key pair.;
        \texttt{<key\_type>} is usually \texttt{rsa};
        \texttt{<key\_length>} is usaully 4096 (in bits).
    }

    \CommandEntry{ssh-keygen -l -f <keyfile>}
    {
        Get the SHA-256 fingerprint of the \texttt{<keyfile>};
        Usually, this command is used to obtain the host pubkey fingerprint during the first connection.;
        \texttt{sudo ssh-keygen -l -f /etc/ssh/<pubkey>}
    }

    \CommandEntry{ssh-copy-id -i <pubkey> <user>@<host>}
    {
        Copy the public key to the SSH server.;
        \texttt{<user>} is the user on the SSH server.;
        \texttt{<host>} is the hostname or IP address of the SSH server.
    }

    \CommandEntry{ss -o state established [ dport | sport ] = :<port \#>}
    {
        Display the established socket whose remote or local port number is \texttt{<port \#>};
        \texttt{dport} is the remote port. \texttt{sport} is the local port.;
        Filters can be combined using \texttt{and}, \texttt{or} For example, \texttt{ss -o state established '( dport = :<port \#1> or sport = :<port \#2> )'};
        \textbf{Example:} \texttt{ss -o state established '( dport = :ssh or sport = :ssh )'} will retrieve all incoming and outgoing SSH connections on the local machine.
    }

    \CommandEntry{sudo ufw status verbose}
    {
        Show current status of the firewall in a verbose manner.
    }

    \CommandEntry{sudo ufw status numbered}
    {
        Show firewall rules with a number assigned to each of them.
    }

    \CommandEntry{sudo ufw enable}
    {
        Enable the firewall.
    }

    \CommandEntry{sudo ufw disable}
    {
        Disable the firewall.
    }

    \CommandEntry{sudo ufw reload}
    {
        Reload the firewall after modify its configuration files.
    }

    \CommandEntry{sudo ufw [ allow | deny | reject ] [ in | out ] < port\_\#/[ tcp | udp ] | protocol\_name >}
    {
        Add a rule with respect to a certain port to the firewall.;
        \texttt{deny} will instruct the firewall to drop packets silently while \texttt{reject} may instruct the firewall to respond with error messages.;
        If direction is not specified, \texttt{in} will be the default value.;
        Available \texttt{<protocol\_name>} can be found in the file \texttt{/etc/services};
        \textbf{Example:} \texttt{sudo ufw allow ssh} is equivalent to \texttt{sudo ufw allow 22/tcp}
    }

\section{Kernel}

    \CommandEntry{sudo dmesg}
    {
        Output messages produced by kernel ring buffer.;
        \texttt{-T}: human-readable timestamps (but inaccurate).;
        \texttt{--follow}: real-time monitoring.;
        \texttt{-l <log\_level\_list>}: filter the output by log levels. \texttt{<log\_level\_list>} can be a comma separated list. For example, \texttt{sudo dmesg -l warn,err} 
    }

\section{Power Source Management}

    \CommandEntry{upower -e}
    {
        Enumerate all power supply devices.;
        Each device is specified by a D-Bus object path. Not the path in the file system.
    }

    \CommandEntry{upower -i <path>}
    {
        Get all information about the power supply device specified by \texttt{<path>};
        The list of \texttt{<path>} can be obtained via \texttt{upower -e}
    }

\section{Git}

    \CommandEntry{git init}
    {
        Create an empty Git repository or reinitialize an existing one.
    }

    \CommandEntry{git config --local user.name "<username>"}
    {
        Set user name as \texttt{<username>} only for the current git repository.;
        If \texttt{--global} is applied, instead of \texttt{--local}, the configuration will be applied globally.
    }

    \CommandEntry{git config --local user.email "<email>"}
    {
        Set user email as \texttt{<email>} only for the current git repository.;
        If \texttt{--global} is applied, instead of \texttt{--local}, the configuration will be applied globally.
    }

    \CommandEntry{git branch <branch\_name>}
    {
        Create a new branch named \texttt{<branch\_name>}, providing that \texttt{<branch\_name>} does not exist.
    }

    \CommandEntry{git branch -a}
    {
        List all local branches and remote branches.
    }

    \CommandEntry{git branch -M <new\_name>}
    {
        Forcibly rename the current branch with the new name \texttt{<new\_name>}
    }

    \CommandEntry{git checkout <branch\_name>}
    {
        Switch to the branch \texttt{<branch\_name>}
    }

    \CommandEntry{git checkout -b <branch\_name>}
    {
        Create a new branch named \texttt{<branch\_name>} and switch to it immediately, providing that \texttt{<branch\_name>} does not exist.;
        Shorthand for \texttt{git branch <branch\_name>} and \texttt{git checkout <branch\_name>}
    }

    \CommandEntry{git checkout <commit\_id>}
    {
        Jump to a specific commit temporarily.;
        \texttt{<commit\_id>} can be retrieved by running \texttt{git log};
        Use \texttt{git checkout <current\_branch>} to come back to the \texttt{HEAD}
    }

    \CommandEntry{git stash}
    {
        Move away all modifications made on top of the \texttt{HEAD} and save them somewhere.;
        Equivalent to \texttt{git stash push}
    }

    \CommandEntry{git stash pop}
    {
        Reverse operation of \texttt{git stash push}
    }

    \CommandEntry{git remote -v}
    {
        Show remote repository's URL after its name.;
        Shorthand for \texttt{git remote --verbose}
    }

    \CommandEntry{git remote add <name> <url>}
    {
        Add a remote repository witn name as \texttt{<name>} at \texttt{<url>};
        \textbf{Example:} \texttt{git remote add origin https://<TOKEN>@github.com/<username>/<repository>.git};
        \textbf{Note:} \texttt{https://<TOKEN>@github.com/<username>/<repository>.git} is a basic authentication URL. With such kind of URL, you don't need to enter your credentials for every push manually. But in this way, your token is stored in a plain text so it can be retrieved easily, whish may introduce additional security risk. Do it at your own risk!
    }

    \CommandEntry{git remote set-url <name> <new\_url>}
    {
        Change the URL of the remote repository \texttt{<name>} to \texttt{<new\_url>};
        \textbf{Example:} \texttt{git remote set-url origin https://<TOKEN>@github.com/<username>/<repository>.git}
    }

    \CommandEntry{git add <file>}
    {
        Add \texttt{file} to the staging area.
    }

    \CommandEntry{git commit -m "<msg>"}
    {
        Commit all work in the staging area to the local repository with commit message \texttt{<msg>};
        Use \texttt{-a} option together to add all modified or deleted files to the staging area. Note that new files haven't been indexed won't be affected.
    }

    \CommandEntry{git push -u <remote\_name> <local\_branch>}
    {
        Set the upstream (push destination) as \texttt{<remote\_name>} for \texttt{<local\_branch>}, and push \texttt{<local\_branch>} to \texttt{<remote\_name>};
        \textbf{Example:} \texttt{git push -u origin main};
        Run \texttt{git push} on the \texttt{main} branch next time will directly push it to \texttt{origin}
    }

    \CommandEntry{git config --local pull.rebase true}
    {
        Use rebase to fix the divergent branch while performing pull operation.;
        \textbf{Use Scenario:} different new changes have committed to the both remote and local branch. New changes committed to the local branch will be based on the new changes committed to the remote branch if rebase is enabled.;
        \textbf{Frequently Seen Command Sequence:} \\
        \texttt{git commit  \# commit new features to the local branch} \\
        \texttt{git push  \# failed due to divergent branch} \\
        \texttt{git config --local pull.rebase true  \# if not configured before} \\
        \texttt{git add <confilcted\_files>} \\
        \texttt{git rebase --continue}
    }

    \CommandEntry{git clone <url>}
    {
        Clone / Download a git repository from \texttt{<url>}
    }

    \CommandEntry{git pull}
    {
        Download and overwrite the current branch from the remote repository.
    }

    \CommandEntry{git merge <src\_branch>}
    {
        Apply all divergent changes made on \texttt{<src\_branch>} to the current branch.;
        A new commit will be made on the current branch to record the merge operation if no conflict occurs.;
        If any conflicts occur, manually resolve all conflicts. Use \texttt{git add} to mark the completion. Use \texttt{git commit} to finalize the merge operation.
    }

    \CommandEntry{git log}
    {
        Retrieve the commit logs of the current branch.
    }

    \CommandEntry{git status}
    {
        Retrieve the status of the current branch.
    }

\section{Misc}

    \CommandEntry{sleep <seconds>}
    {
        Ask the shell script to pause for \texttt{<seconds>}
    }

    \CommandEntry{date}
    {
        Display the current time.;
        \texttt{date +\%y\%m\%d}: output the current time in two-digit year, month, and day format. 
    }

    \CommandEntry{who}
    {
        Display all logged in users.
    }

    \CommandEntry{echo <string>}
    {
        Display \texttt{<string>} to the screen.;
        Quotes are not required by default to delineate the \texttt{<string>};
        However, if \texttt{<string>} contains double quotes, single quotes are required to delineate the \texttt{<string>}, and vice versa.;
        Use \texttt{-n} to suppress the newline at the end of the \texttt{<string>};
        Variables can be placed within the double quotes.
    }

    \CommandEntry{exit <exit\_status>}
    {
        Specify the exit status of a shell script.;
        Can be combined with variables, e.g. \texttt{exit \$var1};
        Be careful, the biggest exit status is 255. If \texttt{<exit\_status> > 255}, \texttt{<exit\_status> \%= 256}
    }

\section{Environment Variables}

    \EnvVarEntry{BASH}
    {
        Full pathname to execute current running bash instance.
    }

    \EnvVarEntry{SHELL}
    {
        Full pathname to the bash shell.
    }

    \EnvVarEntry{SHLVL}
    {
        Shell level of the current running shell.;
        Increment by 1 in a new child shell.
    }

    \EnvVarEntry{BASH\_ENV}
    {
        Used in non-interactive subshell. When set, the bash script specified by this variable will be run first before running any scripts.
    }

    \EnvVarEntry{BASH\_SUBSHELL}
    {
        Number of subshells running.
    }

    \EnvVarEntry{HISTSIZE}
    {
        Number of bash history records that will be kept.
    }

    \EnvVarEntry{HOSTNAME}
    {
        The name of the current host.
    }

    \EnvVarEntry{HOSTTYPE}
    {
        A string describe the machine that the bash session is running on.
    }

    \EnvVarEntry{LINENO}
    {
        Line number of the script currently being executed.
    }

    \EnvVarEntry{SECONDS}
    {
        The number of seconds since shell was started.
    }

    \EnvVarEntry{PWD}
    {
        Current working directory.
    }

    \EnvVarEntry{OLDPWD}
    {
        The previous working directory.
    }

    \EnvVarEntry{PPID}
    {
        Process ID of the current bash shell's parent.
    }

    \EnvVarEntry{\$}
    {
        Process ID of the current bash shell process.
    }

    \EnvVarEntry{USER}
    {
        Current user's username.
    }

    \EnvVarEntry{HOME}
    {
        Current user home directory location.
    }

    \EnvVarEntry{UID}
    {
        User ID of the current user.
    }

    \EnvVarEntry{PATH}
    {
        A colon-separated list where shell looks for commands.
    }

    \EnvVarEntry{PS1}
    {
        The primary prompt string which is displayed before every command.
    }

    \EnvVarEntry{PS2}
    {
        The secondary prompt string which is displayed when a command needs more input (e.g. multi-line commands).
    }

    \EnvVarEntry{RANDOM}
    {
        Returns a random number between 0 to 32767.;
        Assigning a need to this environment variable will seed the pseudo-random number generator.
    }

\section{Important Files}

    \FileEntry{/dev/zero}
    {
        Provides continuous zero bytes as read from it.
    }

    \FileEntry{/dev/null}
    {
        Discard any data output to it.
    }

    \FileEntry{/etc/services}
    {
        List of well-known network protocols with their ports.;
        Protocols listed herein can be used in the firewall rules of UFW.
    }

    \FileEntry{/etc/ssh/sshd\_config}
    {
        SSH server configuration file.
    }

    \FileEntry{/etc/ssh/ssh\_host\_*\_key.pub}
    {
        SSH server public keys provided that the current machine acts as a server.;
        Fingerprints will be used to authenticate the server on the first connecting attempt.
    }

    \FileEntry{/sys/class/power\_supply/}
    {
        \texttt{/sys/class/power\_supply/} contains kernel objects about power supplies.; 
        \texttt{/sys/} is the mounting point of sysfs file system, which exposes kernel objects to the user space.;
        \texttt{/sys/class/} groups the devices by classes.;
        \texttt{/sys/class/power\_supply/BAT0} is a symbolic link to a directory which contains information about battery0.;
        \texttt{/sys/class/power\_supply/BAT0/charge\_control\_end\_threshold} control the threshold stopping to charge the battery0. Changing the threshold will immediately affect the battery.
    }

    \FileEntry{/etc/fstab}
    {
        Filesystem table.;
        Filesystem in the table will be mounted automatically at the boot time.
    }

    \FileEntry{/etc/passwd}
    {
        Insecure information about user accounts.;
        \textbf{Format:} \texttt{<login\_name>:<password>:<UID>:<GID>:<comment\_field>:<home\_dir>:<shell\_program>};
        UID 0 is always reserved for root.;
        \texttt{<password>} is \texttt{x}. The real passwords are stored in \texttt{/etc/shadow};
        UIDs below 500 are reserved for system accounts.;
        You can perform user management by editing this file, but it's very dangerous because corrupted file will even prevent root from login.
    }

    \FileEntry{/etc/group}
    {
        Group information.;
        \textbf{Format:} \texttt{<group\_name>:<group\_password>:<GID>:<users>};
        \texttt{group\_password} allows a non-group user temporarily get the membership of the group.;
        GIDs below 500 are reserved for the system.;
        \texttt{<users>} is a list of users that belong to the group. Some \texttt{<users>} may be empty but it's not the case that the group is empty. If a user specify a group as the default group in \texttt{/etc/passwd} file, then the user won't appear in \texttt{<users>} of that group.;
        Do not edit this file manually. Use \texttt{usermod} instead.
    }

    \FileEntry{/bin/sh}
    {
        A soft link. Indicates the default system shell.;
        The default system shell is used to run system shell scripts.
    }

    \FileEntry{*bin*}
    {
        External / filesystem commands, opposite to built-in commands.;
        \textbf{Examples:} \texttt{/bin}, \texttt{/usr/bin}, \texttt{/sbin}, and \texttt{/usr/sbin}
    }

    \FileEntry{/etc/profile}
    {
        Main startup file for the bash shell when acting as a login shell.;
        It's a bad idea to put global variables needed by all users in this file because this file may change when Linux distributions upgrade.
    }

    \FileEntry{/etc/profile.d}
    {
        The directory to place application-specific startup files at the login time.;
        All files under this directory will be sourced by \texttt{/etc/profile};
        Good place to put global variables needed by all users. Just create a new \texttt{.sh} file and put new variables there.
    }

    \FileEntry{\$HOME/.bashrc}
    {
        User-specific startup file to define user-specific environment variables.;
        If a shell is started as a non-login interative shell, it will only check \texttt{.bashrc} file.;
        \texttt{\$HOME/.profile}, \texttt{\$HOME/.bash\_profile}, \texttt{\$HOME/.bash\_login} are also user-specific startup files, but only some of them will be used.;
        Best place to store user-specific persistent variables for all types of shell process, except that \texttt{BASH\_ENV} is set and points to somewhere else.
    }

    \FileEntry{/etc/apt/sources.list}
    {
        \texttt{apt} repository file.;
        \textbf{Format:} \\
        \texttt{deb <url> <distro\_name> <pkg\_type\_list>} \\
        \texttt{deb-src <url> <distro\_name> <pkg\_type\_list>} \\
        \texttt{deb} indicates the repository contains compiled binaries. \\
        \texttt{deb-src} indicates the repository contains source codes. \\
        \texttt{<pkg\_type\_list>} can contain multiple words.
    }

\section{Command Line Editors}
    
    \subsection{Vim}

        \subsubsection{Installation and Invocation}

            On some distributions, \texttt{vim} is installed and has an alias of \texttt{vi}. Use \texttt{alias vi} to check it.

            However, \texttt{vim} may be preinstalled with a minimal version, \texttt{vim-tiny} on some distributions. To check it, use \texttt{which vi} to find the file path. Then use \texttt{ls -l <path>} to inspect the file type. If the file is a symlink, then it's probably a minimal version. Sometimes, the symlinks are chained. In order to peek the final target, use \texttt{reaklink -f <path>}

            Use \texttt{sudo apt install vim} to install the full version. Use \texttt{vim <file>} to open a file. If \texttt{<file>} does not exist or \texttt{<file>} is omitted, \texttt{vim} will first open a new buffer for editing.

        \subsubsection{Modes of Operation}

            \begin{itemize}
                \item Command Mode (aka Normal Mode)
                \item Insert Mode
                \item Ex Mode
            \end{itemize}

            Press \texttt{i} to enter the insert mode from the command mode.\\
            Press Esc to enter the command mode from the insert mode. \\
            Press \texttt{:} to enter the Ex mode.

        \subsubsection{Common Operations}

            \textbf{Navigation} \\
            \vspace{-1em}
            \begin{itemize}
                \item Arrow keys / \texttt{h, j, k, l}: Move one character.
                \item PageDown (or Ctrl+F), PageUp (or Ctrl+B): Move one page forward, backward.
                \item \texttt{G}: go to the last line.
                \item \texttt{gg}: go to the first line.
                \item \texttt{<\#> G}: go to the \texttt{<\#>}-th line.
            \end{itemize}

            \textbf{Save and Quit}
            \begin{itemize}
                \item \texttt{q}: quit if no changes are made.
                \item \texttt{q!}: quit an discard all changes made.
                \item \texttt{w <file>}: save the content to a new file \texttt{<file>}
                \item \texttt{wq}: save and quit.
            \end{itemize}

            \textbf{Editing Data}
            \begin{itemize}
                \item \texttt{x}: delete the character at the current cursor. \texttt{2x} means delete 2 characters.
                \item \texttt{dd}: cut the current line. \texttt{2dd} means delete 2 lines.
                \item \texttt{dw}: cut the current word.
                \item \texttt{d\$}: cut to the end of the current line from the current cursor.
                \item \texttt{p}: paste the cut or copied contents to the current cursor position.
                \item \texttt{yd}: copy the current line.
                \item \texttt{yw}: copy the current word.
                \item \texttt{y\$}: copy to the end of the current line from the current cursor.
                \item \texttt{v}: enter visual mode to select contents. Usually used before using \texttt{y}   
                \item \texttt{J}: Remove the newline character from the current line (merge two lines).
                \item \texttt{u}: undo the previous operation.
                \item \texttt{a}: append data after the cursor (move the cursor one character forward and enter the insert mode).
                \item \texttt{A}: append data to the end of the current line.
                \item \texttt{r <char>}: replace the character at the cursor with \texttt{<char>}
                \item \texttt{R <data>}: replace the contents from the current cursor position with \texttt{<data>}, until Esc is pressed.
            \end{itemize}
            \textbf{Note}: Press backspace will only move the cursor one character backward. Press delete will delete the character at the current cursor position.

            \vspace{1em}
            \textbf{Searching and Replacing}
            \begin{itemize}
                \item \texttt{/<pattern>}: search for \texttt{<pattern>}. Press \texttt{n} or \texttt{/}+Enter to find next.
                \item \texttt{:s/<old>/<new>/g}: replace \texttt{<old>} with \texttt{<new>} in the current line.
                \item \texttt{:n,ms/<old>/<new>/g}: replace all \texttt{<old>} with \texttt{<new>} from line \texttt{n} to line \texttt{m}
                \item \texttt{:\%s/<old>/<new>/g}: replace all \texttt{<old>} with \texttt{<new>} in the current file.
                \item \texttt{:\%s/<old>/<new>/g}: replace all \texttt{<old>} with \texttt{<new>} in the current file with confirmation for each occurence.
            \end{itemize}

    \subsection{Nano}
        \begin{itemize}
            \item Ctrl+C: Displays the cursor's position.
            \item Ctrl+G: Displays nano's help window.
            \item Ctrl+J: Justifies the current text paragraph.
            \item Ctrl+K: Cuts the text line.
            \item Ctrl+O: Writes out the current text editing buffer to a file.
            \item Ctrl+R: Reads a file into the current text editing buffer.
            \item Ctrl+T: Run other commands.
            \item Ctrl+U: Pastes text stored in cut buffer and places in current line.
            \item Ctrl+V: Got to the next page.
            \item Ctrl+W: Searches for word or phrases within text editing buffer.
            \item Ctrl+X: Closes the current text editing buffer, exits nano, and returns to the shell.
            \item Ctrl+Y: Go to the previous page.
            \item M-U: Undo.
            \item M-E: Redo.
        \end{itemize}
        \textbf{Note:} M refers to the Meta Key, which usually is Alt or Esc.

\section{Shell Scripting}

    \CommandEntry{command1 ; command2 ; \dots ; commandn}
    {
        Chain multiple commands sequentially.;
        Chain as many commands as desired.;
        The maximum command-line character count is 255 characters.;
        This is only needed when you want to put multiple commands into one line.;
        In a shell script, you can put multiple commands into separate lines.
    }

    \CommandEntry{\#!<path\_to\_shell>}
    {
        Specify the shell used to execute the script.;
        The first line of shell scripts.;
    }

    \CommandEntry{\#<comment>}
    {
        Comment line of shell scripts.
    }

    \ScriptEntry{Process of Editing and Running a Shell Script}
    {
        Edit the script.;
        Add execute permission (e.g. \texttt{chmod u+x myscript.sh}).;
        Add the path containing the script to the \texttt{PATH} variable (e.g. \texttt{PATH=\$PATH:.}) or use absolute / relative path (e.g. \texttt{./myscript.sh}) to refer to the script.;
        Run the script.;
        \textbf{Note:} Running a script by referencing the absolute / relative path will invoke a shell as a subprocess instead of a subshell. A \texttt{bash} subprocess is not necessarily to be a subshell.
    }

    \ScriptEntry{Display Special Characters}
    {
        Use back slash (\textbackslash) to escape a special character
    }

    \ScriptEntry{User Variables}
    {
        \textbf{Naming Rules}:
            \begin{itemize}
                \item any text string up to 20 characters.
                \item may contain letters, digits, or underscores.
                \item case sensitive.
            \end{itemize};
        \textbf{Referencing:} \texttt{\$<var\_name>};
        \textbf{Assigning:}
            \begin{itemize}
                \item No space may appear between the variable, equal sign, and the value.
                \item No dollar sign. \texttt{<var\_name>=<value>}
            \end{itemize};
        \textbf{Data Type:} String, by default.;
        \textbf{Life Cycle:} Throughout the life of shell script but are deleted when the shell script completes.
    }

    \ScriptEntry{Command Substitution}
    {
        Assigning the output of a command to a variable.;
        \textbf{Format:}
            \begin{itemize}
                \item \texttt{\`<command>\`}
                \item \texttt{\$(<command>)}
            \end{itemize};
        No space between assignment equal sign and command substitution characters.;
        Using command substitution will invoke a subshell to run the enclosed command. Any variable created in the script won't be accessible to the subshell.
    }

    \ScriptEntry{Output Redirection}
    {
        \textbf{Basic Mode:} \texttt{<command> > <output\_file>};
        If \texttt{<output\_file>} already exists, it will be overwritten.;
        \textbf{Appending Mode:} \texttt{<command> >> <output\_file>}
    }

    \ScriptEntry{Input Redirection}
    {
        \textbf{Basic Mode:} \texttt{<command> < <input\_file>};
        \textbf{Inline Mode:} \\ \texttt{<command> << <marker> \\ <stdin\_data> \\ <marker>}
        \begin{itemize}
            \item \texttt{<marker>} can be any text string, but must be the same.
            \item \texttt{<stdin\_data>} will be prompted by \textbf{the secondary prompt}, which is specified by \texttt{\$PS2}
        \end{itemize}
    }

    \ScriptEntry{Pipe}
    {
        \textbf{Format:} \texttt{<command1> | <command2> | \dots | <commandn>};
        Commands are not run back to back. They are run in the same time.;
        No intermediate file or buffer are used.;
        As soon as the previous command produces data, the next commmand gets busy processing on them.;
        If too many data are produced, use \texttt{more} to paging.;
        Output redirection can be used together with piping. \\ \texttt{<command1> | <command2> | \dots | <commandn> > <output\_file>}
    }

    \ScriptEntry{Math Operations: \texttt{expr}}
    {
        \textbf{Format:} \texttt{expr <expression>};
        For expressions like \texttt{<arg1> <op> <arg2>}, spaces are required between arguments and operators.;
        Many \texttt{<op>} are special characters in Bash, such as \texttt{*}, which need escaping by backslash (\textbackslash).;
        For detailed information about supported \texttt{<expression>}, use \texttt{man expr};
        \textbf{Limitation:} cannot perform floating point operations.
    }

    \ScriptEntry{Math Operations: Brackets}
    {
        \textbf{Format:} \texttt{\$[ <operation> ]};
        Don't need to escape special characters.;
        Spaces between arguments and operators are not required.
    }

    \ScriptEntry{Floating Point Math Operations: \texttt{bc}}
    {
        Use \texttt{bc} to invoke. Use \texttt{bc -q} to invoke quietly.;
        Use \texttt{quit} to exit.;
        Precision is controlled by the built-in variable \texttt{scale}. \texttt{scale=0} by default (i.e. integer mode).;
        Trimming instead of rounding off.;
        Support variables, \texttt{<var\_name>=<value>}. Dollar sign is not required for referencing.;
        Use \texttt{print} to print variables or numbers.;
        Usually combined with command substitution and piping. \\ 
        \texttt{var=\$(echo "<operation1>; <operation2>; \dots; <operationn>" | bc)};
        When many math operations are needed, inline input redirection is the best way to use. \\
        \texttt{var=\$(bc << EOF \\ <operation1> \\ <operation2> \\ \dots \\ <operationn> \\ EOF \\ )};
        Variables created in the \texttt{bc} are only valid in the \texttt{bc} and not accessible to the shell script.
    }

    \ScriptEntry{Exit Status}
    {
        An integer ranging from 0 to 255.;
        0 means success, and a positive exit status means error, by convention.;
        No standard convention for error exit status. See EXIT STATUS section of \texttt{man bash} for more information.;
        Use the variable \texttt{\$?} to check the exit status of the last command.;
        Shell scripts will return the exit status of the last command by default.;
        Use \texttt{exit <exit\_status>} to specify the exit status.
    }

\end{document}
